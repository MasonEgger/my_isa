
\title{A Custom ISA and Emulator}
\author{
        Mason Egger \\
                Undergraduate Computer Science Major\\
                Texas State University\\
}
\date{\today}

\documentclass[12pt]{article}
\usepackage{graphicx}
\graphicspath{ {images/} }

\begin{document}
\maketitle



\section{Introduction}
The purpose of this project was to give a deeper insight to the complexity and design of Instruction Set Architectures.
This design is a derivative of two common and historically significant ISAs. Implementation of this ISA was done in Java,
allowing for the design of different hardware components to be abstracted out into classes. 

\paragraph{Outline}
The remainder of this article is organized as follows.
Section~\ref{design choices} explains the different aspects of this ISA
and where they were derived from. 
Section~\ref{instructions} contains a list of all supported instructions. 
Section~\ref{sample code} contains code used to test all functionality of the ISA and Emulator.
Section~\ref{results} shows output of the run of the sample code. The output contains a graphical represenation of the registers as they are being updated. 
Finally, Section~\ref{conclusions} summarizes the project and conlusions drawn.

\section{Design Choices}\label{design choices}
This design is based heavily on two very important RISC based ISAs, MIPS and z80. The choice to combine these two was inspired by the register design 
of z80 (which eventually led to the current design in x86) as well as the fixed instruction length of MIPS. As seen in the figure below the z80 
register design allows for register combining to allow for larger variables to be stored, but not waste the entire address space on a value that 
could be done in a fraction of the space. This ISA does currently support register combinations. 

\centerline{\includegraphics{registers}}

\section{Instructions}\label{instructions}

\centerline{\textbf{The supported instructions are listed below.}}

The instructions were designed according to the following opcode. In the instance where not all operands were needed, 0s were placed substituted to 
allow for fixed length instructions\\

\begin{tabular}{c | c | c | c}
\textbf{instruction} & \textbf{storage} & \textbf{register 1} & \textbf{register 2} \\
\end{tabular}


\begin{tabular}{ r | l } 
add A B C & Adds B and C  \\
sub A B C & Subtracts the values in registers B and C and stores into register A \\
mul A B C & Multiplys the values in registers B and C and stores into register A \\
div A B C & Divides the values in registers B and C and stores into register A \\
ldi A 1 0 & Load Immediate, Loads 1 into A. \\
ld A B 0 & Load, Loads into A what is at the memory address stored in B \\
st A B 0 & Store, Stores the value of A into the memory address stored in B \\
in A 0 0 & In, user input stored into register A \\ 
out A 0 0 & Out, output value in register A \\
push A 0 0 & Push. Pushes A onto the stack \\
pop A 0 0 & Pop. Pops of the stack into A \\
j A 0 0 & Jump. \\
beq A B C & Jump if equal. \\
bne A B C & Jump if not equal. \\
bgt A B C & Jump if greater than. \\
bge A B C & Jump if greater than or equal. \\
blt A B C & Jump if less than. \\
ble A B C & Jump if less than or equal. \\
inc A 0 0 & Increment the value of A by 1 \\
dec A 0 0 & Decrement the value of A by 1 \\
\end{tabular}

Jumps increment the program counter by the value of A (A may be + or -)\\
Jump comparisons are done for the values of B and C\\

\section{Sample Code}\label{sample code}
This simple program has the user input two numbers. If the first number is less than the second number the program adds the value of the first number 
to a sum, increments the first value by 1, and continutes until the first value is equal to the second value. \\
\begin{tabular} {l | r}
in A 0 0 & Stores a user inputed value into register A \\
in B 0 0 & Stores a user inputed value into register B \\
ldi C 4 0 & Loads the immediate value of 4 into C for branch control \\
ldi L -3 0 & Loads the immediate value of -3 into L for branch control \\
bge C A B & Branch if A is greater than or equal to B \\
add D D A & Add into register D the sum of D and A \\
inc A 0 0 & Increments the value of A \\
j L 0 0 & Branch - adjust program counter to sum of PC  and L \\
out D 0 0 & Output the value of D \\
\end{tabular}

\section{Results}\label{results}
The results of the above program are displayed in the accompanying text file named results.txt \\


\section{Conclusions}\label{conclusions}
Through this project greater insight was achieved on the complexities of design and style of Instruction Set Architectures. Large quantities of time 
were spent on this design and implementation which led to success for the project. This project has a working emulator for a unique ISA designed by 
the author. While there is still much more that could be implemented, this project is a success. 


\end{document}
